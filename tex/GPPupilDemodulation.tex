\documentclass[a4paper,11pt,twoside]{scrartcl}
\input{preamble}

\usepackage[utf8]{inputenc}
\usepackage[T1]{fontenc}
\usepackage{stmaryrd}
\RequirePackage[hyperref,table]{xcolor}



\renewcommand{\AdjointLetter}{\mathrm{T}}
\usepackage{lmodern}
\DeclareUnicodeCharacter{00B5}{\ensuremath{\mu}} % µ
\DeclareUnicodeCharacter{2264}{\ensuremath{\le}} % ≤
\DeclareUnicodeCharacter{2265}{\ensuremath{\ge}} % ≥
\DeclareUnicodeCharacter{22C5}{\!\cdot\!}
\usepackage{graphicx} % Required for inserting images


\title{Notes about GRAVITY+ metrology demodulation}


\author{Ferréol Soulez }
\date{February 2023}

\begin{document}

\maketitle

\section{The metrology table}
The metrology data is stored as a FITS Table in the \verb+METROLOGY+ HDU (10$^\textrm{th}$) as in table \ref{tab:table}. The voltages \verb+VOLT+ are composed of $80$ columns:
\begin{itemize}
    \item 2 directions ($x$ and  $y$) per diodes
    \item 4 diodes per telescope (one on each spiders)
    \item 1 fiber coupler diode (labeled \verb+FC+) per telescope
    \item 4 telescopes
    \item 2 sides: \verb|FT|  and \verb|SC|
\end{itemize}
The signals (the table column) are sample at $500\,$Hz leading to a very large number of rows. 
\begin{table}[]
    \centering
    \begin{tabular}{l| l ll}
Name   &       Size  & Type  &   TFORM   \\
\hline
TIME   &              & Int32  &  1J     \\
VOLT   &       (80,) & Float32 &  80E    \\
POWER\_LASER   &    &   Float32 & 1E     \\
LAMBDA\_LASER   &    &  Float32 & 1E     
    \end{tabular}
    \caption{Metrology table}
    \label{tab:table}
\end{table}

\section{The modulation model}

When the pupil modulation is on (\verb|ESO INS PMC1 MODULATE| keyword is \verb|true|), the metrology signal is modulated at a frequency of $f=1\,$Hz. 
This modulation does not affect the fiber coupler diode.

For each diode in \{D1, D2, D3, D4\}, we define the complex measurement $\V{v} \in \Complexes^N$  with $v_i =   x_i + \jmath y_i \,, \quad \forall i \in \llbracket 1,N \rrbracket $ and  $N$ is the number of row of the metrology. The modulation as a function of time is:
\begin{equation}
    \V{v} = \exp\left( \jmath \, b \sin\left( \V{\omega} + \phi\right)\right)\,.\label{eq:v}
\end{equation}
where $\omega = 2\,\pi\,\V{t}$ is the modulation pulsation built from the \verb|TIME| column, the amplitude $b$  and the phase $\phi$  are the modulation parameters to be estimated to demodulate the metrology signal.


\section{The overall model}
For a  diode the measured modulated voltage data $\V{d}$ can be modeled by:
\begin{equation}
    \V{d} = \left( c + \V{s} \times  \exp\left( \jmath \, b \sin\left( \V{\omega} + \phi\right)\right)\right)\times\exp\left(\jmath\,\Phi_\textrm{FC}\right)+  \V{e}\,,\label{eq:model}
\end{equation}
where $\times$ is the element-wise multiplication, $\V{s}$ is the sought-after demodulated  metrology signal, $\Phi_\textrm{FC}$ is the phase of the fiber coupler measurement accounting for FDDL movements. $c  = x_0 + \jmath\,y_0$ is the center of the pupil and $\V{e}$ is a vector representing the errors on the measurement that are supposed Gaussian centered, independent and identically distributed.




\section{How it is actually coded in the pipeline}

\subsection{Processing in chunks}
The demodulation signal is split in chunks of  \verb+MAX_SECONDS_PER_CHUNK = 100+ seconds. In each of these chunks, the phase of the fiber coupler $\Phi_\textrm{FC}$ is supposed to be constant. As a consequence, it is averaged out and each chunk is demodulated independently. 

\subsection{Centering}
Before processing, metrology signal is centered (such that $\V{c}=\V{0}$). If the keyword \verb+gravity.metrology.use-dark-offsets=TRUE+  it is centered using the dark file if it exists, otherwise, it use hard-coded diode centers computed by S. Gillessen and set in the array \verb+diode_zeros+ in the file \verb|gravi_demodulate.c|. 

\subsection{Averaging}
Within each chunk the metrology is averaged on a single second such that:
\begin{equation}
    \bar{d}_k = \Avg{d_{k+i\,t}, \forall i\in \Natural}\,,
\end{equation}
where $t$ is the number of steps in 1 second given by the hard-coded constant \verb|STEPS_PER_SECOND=500|.
The model of the metrology described equation~\ref{eq:model} is rewritten as:
\begin{equation}
    \V{\bar{d}}' = a \times  \exp\left( \jmath \, b \sin\left( \V{\bar{\omega}} + \phi'\right) + \phi\right)+  \V{e}\,,\label{eq:modelpip}
\end{equation}
where $\bar{\omega}_k = k\,\frac{2\,\pi}{500}$.

The pipeline code use the variables \verb+a, b, pha1, pha2+ where \verb|pha1|$=\phi'$, \verb|pha2|$=\phi$ and $s = a \exp(\jmath\,\phi)$.


\subsection{Parameters fitting}
For each chunk, the parameters \verb+a, b, pha1, pha2+ are estimated using the Nelder-Mead Simplex algorithm minimizing the chi-square $\chi^2$ :
\begin{equation}
    \chi^2 \left( a, b, \phi, \phi2 \right) =   \Norm{\V{d}' -  a \times  \exp\left( \jmath \, b \sin\left( \V{\omega} + \phi' \right) + \phi \right)}
\end{equation}
To ensure global minimum, this minimization is performed multiple times with different initializations:
\begin{itemize}
    \item $a_0 = \mathrm{std}(\V{d}')$
    \item $b_0 = 0.25$
    \item $\phi_0 = \{-\pi/2, 0\} $
    \item $\phi'_0 = \{-\pi/2, 0\} $
\end{itemize}
The phases are then unwrapped if needed.

\subsection{Computation of the modulation}
Once the parameters $a,\,b,\,,\phi,\,\phi'$ have been estimated for each chunk, the modulation $\psi$ is computed over a second:
\begin{equation}
   \psi_k =  -\arctan(d_k) - \phi
\end{equation}

\subsection{Demodulation}
The metrology signal is finally demodulated, chunk-wise each demodulated sample $p_i$  being:
\begin{equation}
   p_i = d_i  \exp\left(- \jmath \,\Psi_{\textrm{mod}(i,500)} \right) 
\end{equation}


\section{ Estimating the modulation parameters}

To estimate the modulation parameters we need to disentangle it from the signal  $\V{s}$. This is done using the expansion:
\begin{equation}
    \V{s} = a  + \V{\delta v}\,
\end{equation}
$\abs{a}$  and $\arg{(a)}$ being the  mean amplitude and phase respectively of the metrology signal.  $ \V{\delta v}$ is centered ($\Expect{\delta v}= 0$)  and supposed to be Gaussian independent and identically distributed.

Under this assumption, the modulation parameters $( b, \phi)$, the center $c$ and the mean of the metrology signal $a$ can be estimated using least square:
\begin{equation}
    (c^+, a^+, b^+, \phi^+) = \argmin_{c,a,b,\phi} \Norm{\V{d} -  \left(c + a \exp\left( \jmath \, b \sin\left( \V{\omega} + \phi\right)\right)\right)\times\exp\left(\jmath\,\Phi_\textrm{FC}\right) }_2^2
\end{equation}
that is equivalent with : 
\begin{equation}
    (c^+, a^+, b^+, \phi^+) = \argmin_{c,a,b,\phi} \Norm{\V{r} -  c - a \exp\left( \jmath \, b \sin\left( \V{\omega} + \phi\right)\right) }_2^2\label{eq:r}
\end{equation}
with $\V{r} =\V{d}\times\exp\left(-\jmath\,\Phi_\textrm{FC}\right)$.

\subsection{Linear estimates}
From the equation \ref{eq:r}, we can see that the parameters  $c$ and $a$ depends linearly of the  $\V{d}$ and the modulation phasor $\V{m} = \exp\left( \jmath \, b \sin\left( \V{\omega} + \phi\right)\right)$. For a given values of $b$ and $\phi$, the optimal values $c^+$ and $a^+$ have a closed-form solution that is:
\begin{equation}
    \left[\begin{array}{l}
    c^+\\
    a^+
    \end{array}\right] = \M{A}^{-1} 
    \left[\begin{array}{l}
    \sum_i d_i\\
    \V{m}\T\V{d}
    \end{array}\right] 
\end{equation}
where $\V{m}\T$ is the conjugate transpose of $\V{m}$ and $\M{A}$ is the $2\times2$ matrix:
\begin{equation}
    \M{A} = \left[\begin{array}{ll}
    N & \sum_i m_i\\
    \sum_i  m_i^* & N
    \end{array}\right] \,,
\end{equation}
where $m_i^*$ is the complex conjugate of $m_i$ and $N$  the number of measurements. Its inverse is:
\begin{equation}
   \M{A}^{-1} =  \frac{1}{N^2 \left(1 - \Abs{ \sum_i d_i}^2\right)}\,\left[\begin{array}{ll}
    N & - \sum_i m_i\\
    - \sum_i  m_i^* & N
    \end{array}\right] 
\end{equation}
For a couple $(b,\phi)$ the optimal $c^+(b,\phi)$ and $a^+(b,\phi)$ are given by:
\begin{align}
    c^+(b,\phi) &= \frac{N\,\sum_i d_i - \V{m}\T\V{d}\sum_i m_i}{N^2 \left(1 - \Abs{ \sum_i d_i}^2\right)}\\
    a^+(b,\phi) &= \frac{ -N\,\left(\sum_i d_i\right)  \left(\sum_i  m_i^*\right)+ \V{m}\T\V{d}}{N^2 \left(1 - \Abs{ \sum_i d_i}^2\right)}
\end{align}

\subsection{Non-linear estimates}
The modulation parameters estimation amounts to estimate only $b$  and $\phi$, optimizing the function:
\begin{equation}
   f(b,\phi) = \Norm{\V{r} -  c^+(b,\phi) - a^+(b,\phi) \exp\left( \jmath \, b \sin\left( \V{\omega} + \phi\right)\right) }_2^2\label{eq:nl}
\end{equation}
\begin{figure}
    \centering
    \includegraphics[width=0.45\linewidth]{figs/GV1D1SC.png}
    \includegraphics[width=0.45\linewidth]{figs/GV3D3SC.png}
    \caption{$f(b,\phi)$ for two different diodes}
    \label{fig:GV1D1SC}
\end{figure}

This function is non-linear and non-convex as it can be seen on figure \ref{fig:GV1D1SC}. The two main optima are equivalent as $f(b,\phi) =  f(-b,\phi+ \pi )$. An initialization with $b$ sufficiently small seems to ensure to end in the global optimum. Note that the case $b=0$ is singular  as $f(b,\phi)$ is equal to the variance of $\V{r}$ in this case whatever is $\phi$  and $ c$.

In the code, this function is minimized by the means of the derivative free NEWOA method of Powell that seems faster than VMLMB (with derivative) and  the Simplex method.
\section{Demodulation}
Once the modulation parameters are estimated the demodulated signal is given by:
\begin{equation}
\V{s} =\left(\V{r} - c^+\right)\times\exp\left(- \jmath \, b^+ \sin\left( \V{\omega} + \phi^+_1\right)\right)\times\exp\left(\jmath\,\Phi_\textrm{FC}\right)\,,\label{eq:demodulation}
\end{equation}

All the $6\times4\times4\times2$ parameters are stored in keywords of the METROLOGY table header as shown on the table \ref{tab:keyword}. These keywords are suffixed with the side, the telescope number and the diode (\eg \verb|DEMODULATION CENTER X0 SC T4 D1|). In addition,  the keyword \verb|PROCSOFT| of the METROLOGY table header is set to \texttt{GPPupilDemodulation.jl}
\begin{table}[]
    \centering
    \begin{tabular}{l c l}
Name   &          & keyword   \\
center    &      $\Re(c^+)$        &   \verb|DEMODULATION CENTER X0|   \\
center   &      $\Im(c^+) $       &   \verb|DEMODULATION CENTER Y0|   \\
metrology mean amplitude  &      $\Abs{a^+} $       &   \verb|DEMODULATION AMPLITUDE ABS|   \\
metrology mean phase  &      $\arg{(a^+)} $       &   \verb|DEMODULATION AMPLITUDE ARG|   \\
modulation amplitude  &      $b^+ $       &   \verb|DEMODULATION SIN AMPLITUDE |   \\
modulation phase  &      $\phi^+ $       &   \verb|DEMODULATION SIN PHASE|   
    \end{tabular}
    \caption{New keywords in the metrology table}
    \label{tab:keyword}
\end{table}

\section{Faint model}

\begin{table}
    \centering
    \begin{tabular}{l c  l}
        Keywords                    & unit  &    \\
        \hline
    \verb|ESO INS ANLO3 RATE1|      & second    &  Rate of repetition of the first state \\     
    \verb|ESO INS ANLO3 RATE2|      & second    &  Rate of repetition of the second state \\	
    \verb|ESO INS ANLO3 REPEAT1|    & Int       &  Number of repetition of the first state\\ 
    \verb|ESO INS ANLO3 REPEAT2|    & Int       &  Number of repetition of the second state\\
    \verb|ESO INS ANLO3 TIMER1|     & second (unix time) &  Starting time of the first state   \\
    \verb|ESO INS ANLO3 TIMER2|     & second (unix time) &  Starting time of the second state \\
	\verb|ESO INS ANLO3 VOLTAGE1|   & V         &  Voltage of the first state \\
    \verb|ESO INS ANLO3 VOLTAGE2|   & V         &  Voltage of the second state
    \end{tabular}
    \caption{Keywords of the FAINT mode}
    \label{tab:faint}
\end{table}

The FAINT mode of the metrology is active when the keyword \verb|ESO INS MET MODE| is set to \verb|FAINT|. In this case metrology laser is dimmed (the LOW state) during observations and maximized between the frame (the HIGH state) according to the keyword shown on table~\ref{tab:faint}.  This alternating HIGH/LOW state is shown on figure~\ref{fig:faint}. In the firsts and the lasts second the metrology power is in NORMAL state.

We define the laser power $p_k$ as the expected laser power at row $k$ for the state $\textrm{ST}(k)$. This vector is filled with   mean of the modulus of the data $\V{d}$  for each state (NORMAL, LOW, HIGH):
\begin{equation}
    p_k = \Avg{\Abs{d_m}}_{m\in \textrm{ST}(k)}
\end{equation}
In addition we consider that the variance of the noise $\V{e}$ varies with the power of the metrology laser. We define the precision (\ie inverse variance) of a measurements at row $k$ as:
\begin{equation}
    w_k = 1/\Var\left(\Abs{d_m}\right)_{m\in \textrm{ST}(k)}
\end{equation}
The noise $\V{e}$ is probably Gaussian with identical variance for both real and imaginary part. As consequence, the variance of its modulus follows a Rayleigh distribution but as we treat equally all states it should not make much difference to assume Normal distribution.

\subsection{Lag on voltage switching}
\begin{figure}
    \centering
    \includegraphics[width=0.8\linewidth]{figs/FaintStates.pdf}
    \caption{Modulus of the T1 FT fiber coupler diode and the state of the metrology laser during a whole observation in FAINT mode.}
    \label{fig:faint}
\end{figure}


As seen on figure \ref{fig:faintzoom}, there is a lag of about 10ms between the switching of voltage and the actual dimming/brightening of the laser. This lag can be automatically estimated and corrected in the Julia code (not by default). In addition, there is also a 10ms transient before the laser power stabilizes. These lags and transients are handle by pre/post-switching delays.
\begin{figure}
    \centering
    \includegraphics[width=0.6\linewidth]{figs/FaintStatesZoom.pdf}
    \caption{Zoom on the modulus of the T1 FT fiber coupler diode and the state of the metrology laser when the laser is switching to HIGH.}
    \label{fig:faintzoom}
\end{figure}

\subsection{Faint model (guess 1)}
In this mode we consider that the laser power affects both  $c$ and the modulated metrology the equation \ref{eq:model} then rewritten as:
\begin{equation}
    \V{d} =  \V{p} \times  \left( c +\V{s} \times  \exp\left( \jmath \, b \sin\left( \V{\omega} + \phi\right)\right)\right)\times\exp\left(\jmath\,\Phi_\textrm{FC}\right)+  \V{e}\,,\label{eq:modelfaint}
\end{equation}
In that case, the equation~\ref{eq:r} solved to retrieve the modulation parameters can be updated   as:
\begin{align}
    (c^+, a^+, b^+, \phi^+) &= \argmin_{c,a,b,\phi} \Norm{\V{r} -  \V{p} \times\left(c +  a \exp\left( \jmath \, b \sin\left( \V{\omega} + \phi\right)\right)\right) }_2^2
\end{align}
To take into an account the precision of the noise, we can rewrite this equation as:

\begin{align}
    (c^+, a^+, b^+, \phi^+) &= \argmin_{c,a,b,\phi} \Norm{\V{r} -  \V{p} \times\left(c +  a \exp\left( \jmath \, b \sin\left( \V{\omega} + \phi\right)\right)\right) }_{\V{w}}^2
\end{align}
with the weighted norm $\Norm{\V x}_{\V w}^2 = \V x\T \,\diag(\V w)\, \V x = \sum_k w_k \Abs{x_k}^2$. This can be also rewritten as:
\begin{align}
    (c^+, a^+, b^+, \phi^+) &= \argmin_{c,a,b,\phi} \Norm{\V{r} \div \V{p}  - \left(c +  a \exp\left( \jmath \, b \sin\left( \V{\omega} + \phi\right)\right)\right) }_{\V w'}^2\label{eq:rfaint}\,,
\end{align}
where $\div$ is the element-wise division and  $   \V w' =  \V w \times \V p \times \V p$

The closed-form solution for  becomes:

\begin{equation}
    \left[\begin{array}{l}
    c^+\\
    a^+
    \end{array}\right] =\left[\begin{array}{ll}
        \sum_i w'_i & \sum_i w'_i\, m_i\\
        \sum_i   w'_i\,m_i^* &  \sum_i w'_i\, \abs{m_i}^2
        \end{array}\right]^{-1} \cdot
    \left[\begin{array}{l}
    \sum_i w'_i\,d_i\\
    \sum_i w'_i\,m_i^*\,d_i\\
    \end{array}\right] 
\end{equation}


\subsection{Faint model (guess 2)}
In this mode we consider that the laser power affects both only the modulated metrology and not the $c$ component. The equation \ref{eq:model} is then rewritten as:
\begin{equation}
    \V{d} =   c +\V{p} \times \V{s} \times  \exp\left( \jmath \, b \sin\left( \V{\omega} + \phi\right)\right)\times\exp\left(\jmath\,\Phi_\textrm{FC}\right)+  \V{e}\,,\label{eq:modelfaint}
\end{equation}
In that case, the equation~\ref{eq:r} solved to retrieve the modulation parameters can be updated   as:
\begin{align}
    (c^+, a^+, b^+, \phi^+) &= \argmin_{c,a,b,\phi} \Norm{\V{d} -  c - a\, \V{p}' \times  \exp\left( \jmath \, b \sin\left( \V{\omega} + \phi\right)\right) }_{\V{w}}^2\,,
\end{align}
where $\V{p}' = \V{p}\times\exp\left(\jmath\,\Phi_\textrm{FC}\right)$.

The closed-form solution for $(c^+, a^+)$  becomes:
\begin{equation}
    \left[\begin{array}{l}
    c^+\\
    a^+
    \end{array}\right] =\left[\begin{array}{ll}
        \sum_i w_i & \sum_i w_i\, \, g_i\\
        \sum_i   w_i\, g_i^* &  \sum_i w_i\, \abs{g_i}^2
        \end{array}\right]^{-1} \cdot
    \left[\begin{array}{l}
    \sum_i w_i\,d_i\\
    \sum_i w'i\, g_i^*\,d_i\\
    \end{array}\right] 
\end{equation}
with $\V{g} =   \V{p}'  \exp\left( \jmath \, b \sin\left( \V{\omega} + \phi\right)\right)$



An example on demodulated phase signal is shown on figure~\ref{fig:FaintDemodulated}.
\begin{figure}
    \centering
    \includegraphics[width=0.8\linewidth]{figs/FaintDemodulated.pdf}
    \caption{Zoom around a HIGH state of the phase of the diode D1 of T1 SC and its demodulation.}
    \label{fig:FaintDemodulated}
\end{figure}
\end{document}
